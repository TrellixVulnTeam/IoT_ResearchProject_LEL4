\documentclass[11pt,]{article}
\usepackage{lmodern}
\usepackage{amssymb,amsmath}
\usepackage{ifxetex,ifluatex}
\usepackage{fixltx2e} % provides \textsubscript
\ifnum 0\ifxetex 1\fi\ifluatex 1\fi=0 % if pdftex
  \usepackage[T1]{fontenc}
  \usepackage[utf8]{inputenc}
\else % if luatex or xelatex
  \ifxetex
    \usepackage{mathspec}
  \else
    \usepackage{fontspec}
  \fi
  \defaultfontfeatures{Ligatures=TeX,Scale=MatchLowercase}
\fi
% use upquote if available, for straight quotes in verbatim environments
\IfFileExists{upquote.sty}{\usepackage{upquote}}{}
% use microtype if available
\IfFileExists{microtype.sty}{%
\usepackage{microtype}
\UseMicrotypeSet[protrusion]{basicmath} % disable protrusion for tt fonts
}{}
\usepackage[margin=0.79in]{geometry}
\usepackage{hyperref}
\hypersetup{unicode=true,
            pdftitle={A Model for Energy-Saving in an IoT Smarthome accounting for End-User Convenience},
            pdfauthor={Alistair Francis Bowman Grevis-James},
            pdfborder={0 0 0},
            breaklinks=true}
\urlstyle{same}  % don't use monospace font for urls
\usepackage{graphicx,grffile}
\makeatletter
\def\maxwidth{\ifdim\Gin@nat@width>\linewidth\linewidth\else\Gin@nat@width\fi}
\def\maxheight{\ifdim\Gin@nat@height>\textheight\textheight\else\Gin@nat@height\fi}
\makeatother
% Scale images if necessary, so that they will not overflow the page
% margins by default, and it is still possible to overwrite the defaults
% using explicit options in \includegraphics[width, height, ...]{}
\setkeys{Gin}{width=\maxwidth,height=\maxheight,keepaspectratio}
\IfFileExists{parskip.sty}{%
\usepackage{parskip}
}{% else
\setlength{\parindent}{0pt}
\setlength{\parskip}{6pt plus 2pt minus 1pt}
}
\setlength{\emergencystretch}{3em}  % prevent overfull lines
\providecommand{\tightlist}{%
  \setlength{\itemsep}{0pt}\setlength{\parskip}{0pt}}
\setcounter{secnumdepth}{5}
% Redefines (sub)paragraphs to behave more like sections
\ifx\paragraph\undefined\else
\let\oldparagraph\paragraph
\renewcommand{\paragraph}[1]{\oldparagraph{#1}\mbox{}}
\fi
\ifx\subparagraph\undefined\else
\let\oldsubparagraph\subparagraph
\renewcommand{\subparagraph}[1]{\oldsubparagraph{#1}\mbox{}}
\fi

%%% Use protect on footnotes to avoid problems with footnotes in titles
\let\rmarkdownfootnote\footnote%
\def\footnote{\protect\rmarkdownfootnote}

%%% Change title format to be more compact
\usepackage{titling}

% Create subtitle command for use in maketitle
\providecommand{\subtitle}[1]{
  \posttitle{
    \begin{center}\large#1\end{center}
    }
}

\setlength{\droptitle}{-2em}

  \title{A Model for Energy-Saving in an IoT Smarthome accounting for End-User
Convenience}
    \pretitle{\vspace{\droptitle}\centering\huge}
  \posttitle{\par}
    \author{Alistair Francis Bowman Grevis-James}
    \preauthor{\centering\large\emph}
  \postauthor{\par}
      \predate{\centering\large\emph}
  \postdate{\par}
    \date{November 2019}

\usepackage{booktabs}
\usepackage{longtable}
\usepackage{array}
\usepackage{multirow}
\usepackage[table]{xcolor}
\usepackage{wrapfig}
\usepackage{float}
\usepackage{colortbl}
\usepackage{pdflscape}
\usepackage{tabu}
\usepackage{threeparttable}
\usepackage{threeparttablex}
\usepackage[normalem]{ulem}
\usepackage{makecell}

\usepackage[ruled,vlined,linesnumbered]{algorithm2e} \SetAlFnt{\tiny} \usepackage{booktabs} \usepackage{longtable} \usepackage{array} \usepackage{multirow} \usepackage[table]{xcolor} \usepackage{wrapfig} \usepackage{float} \floatplacement{figure}{H} \usepackage{caption, setspace} \captionsetup[figure]{font={stretch=1,scriptsize}} \captionsetup[figure]{skip=2pt} \captionsetup[table]{font={stretch=1,scriptsize}} \captionsetup[table]{skip=2pt}

\begin{document}
\maketitle
\begin{abstract}
The preface pretty much says it all.

\par

Second paragraph of abstract starts here. \pagebreak
\end{abstract}

{
\setcounter{tocdepth}{3}
\tableofcontents
}
\hypertarget{algorithm-1}{%
\paragraph{Algorithm 1}\label{algorithm-1}}

Text

\begin{algorithm}[H]
\DontPrintSemicolon
\SetAlgoLined
\KwResult{Intermediate dataframe with 'activity', 'date', 'startTime', 'endTime' as attributes}
\SetKwInOut{Input}{Input}\SetKwInOut{Output}{Output}
\Input{S1 Activities}
\Output{S1 Activities Intermediate}
\BlankLine
\Begin{
  array1-21 = [ ]\;
  
  \While{i < length(dataframe)}{
    array1.append(dataframe[i][0])\;
    array2.append([x.strip() for x in array1[i].split(',')])\;
    array3.append(array2[i][0])\;
    array4.append(array2[i][1])\;
    array5.append(array2[i][2])\;
    array6.append(array2[i][3])\;
    i = i + 1\;
    }
  \While{i < length(dataframe)}{
    array7.append(dataframe[i][1])\;
    array8.append([x.strip() for x in array7[i].split(',')])\;
    array9.append(dataframe[i][2])\;
    array10.append([x.strip() for x in array9[i].split(',')])\;
    array11.append(dataframe[i][3])\;
    array12.append([x.strip() for x in array11[i].split(',')])\;
    array13.append(dataframe[i][4])\;
    array14.append([x.strip() for x in array13[i].split(',')])\;
    i = i + 1\;  
    }
  \While{i < length(dataframe)}{
    for x in range(len(array8[i])) : array15.append(array4[i])\;
    i = i + 1\;  
    }
  for sublist in array8: for item in sublist: array16.append(item)\;
  for sublist in array10: for item in sublist: array17.append(item)\;
  for sublist in array12: for item in sublist: array18.append(item)\;
  for sublist in array13: for item in sublist: array19.append(item)\;
  dfIntermediate = pandas.DataFrame(list(zip(array16, array17, array15, array18, array19)))\;
  start = (dfIntermediate.date + " " + dfIntermediate.startTime)\;
  end = (dfIntermediate.date + " " + dfIntermediate.endTime)\;
  \While{i < length(start)}{
    array20.append(datetime.strptime(start[i], mm/dd/yyyy HH:MM:SS))\;
    array21.append(datetime.strptime(end[i], mm/dd/yyyy HH:MM:SS))\;
    i = i + 1\;
    }
  dfFinal = pandas.DataFrame(list(zip(array16, array17, array20, array21)))\;
  \KwRet dfIntermediate\;
  \KwRet dfFinal\;
}
\caption{Extraction of data from S1 Activities dataset}
\end{algorithm}

\%algorithm usage

\begin{algorithm}
    \KwIn{entities; actors; player; customRules;}
    \KwOut{new world state}
    let \textit{entities} be all entities, excluding actors and the player\;
    let \textit{actors} be all actors, including the player\;
    \ForEach{actor in actors}{
        actor.rules.Invoke()\;
    }
    \tcc{It's possible that one of the rules gave the player a constraint, so we'll check}
    \If{player.constraint == null}{
        GenerateConstraint(player)\;
    }
    \ForEach{rule in customRules}{
        rule.Invoke()\;
    }
    \caption{Step planning}\label{alg:step_planning}
\end{algorithm}

\%procedure usage

\begin{procedure}
    \KwIn{player}
    \KwOut{new constraint for the player}
        \eIf{rules.count $\geq$ 0}{
            \ForEach{item in rules}{
                rules.invoke(player)\;
            }}{
                \textbf{select random} \textit{e} \textbf{from} \textit{entities}\;
                player.constraint = \textit{e}\;
            }
    \caption{GenerateConstraint(Player)}\label{proc:generate_constraint}
\end{procedure}

\begin{algorithm}[htp]
  \SetAlgoLined\DontPrintSemicolon
  \SetKwFunction{algo}{algo}\SetKwFunction{proc}{proc}
  \SetKwProg{myalg}{Algorithm}{}{}
  \myalg{\algo{}}{
  \nl xxx\;
  \nl xxx\;
  \nl \proc{}\;
  \nl \KwRet\;}{}
  \setcounter{AlgoLine}{0}
  \SetKwProg{myproc}{Procedure}{}{}
  \myproc{\proc{}}{
  \nl xxx\;
  \nl \KwRet\;}
  \caption{Algorithm with procedure}
\end{algorithm}


\end{document}

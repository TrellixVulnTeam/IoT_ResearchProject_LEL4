\documentclass[11pt,a4paper,twoside]{article}

% Put your name and thesis title in the next two lines
\newcommand{\student}{W. Mansfield}
\newcommand{\topic}{Enhancing Commitment Machines}

\usepackage{times}
\usepackage{a4wide}
\usepackage{fancyheadings}

%\usepackage{apalike}
\usepackage{natbib} % various citation commands

%% New 
\usepackage{epsfig}
\usepackage{pifont}
\usepackage{fancybox}
\usepackage{amssymb}

\newcommand{\yes}{\ding{52}}
\newcommand{\no}{\ding{56}}
\newcommand{\gap}{\hspace*{1cm}}


\renewcommand{\leftmark}{\student}
\renewcommand{\rightmark}{\topic}
\setlength{\headrulewidth}{0pt}

\begin{document}

\title{{\sc \topic}}
\author{{\sc \student}
% Delete one of the two next lines
	\\[3mm] {\bf Supervisor:} {\sc H. Jacobson}
	\\[5mm] Honours Thesis
	\\[5mm] School of Computer Science and Information Technology
	\\ RMIT University 
	\\ Melbourne, AUSTRALIA
} 
\date{July, 2004}

\maketitle

\begin{abstract}
Agent interaction protocols are usually specified in terms of permissible
sequences of messages. This representation is, unfortunately, brittle and does
not allow for flexibility and robustness.  The {\em commitment machines}
framework of Yolum and Singh aims to provide more flexibility and robustness by
defining interactions in terms of the commitments of agents.  In this thesis we
identify a number of areas where the commitment machines framework needs
improvement and propose an improved version.
In particular we improve the
way in which commitments are discharged and the way in which pre-conditions are specified. 
\end{abstract}


\newpage

\tableofcontents

\newpage

\pagestyle{fancy}


\newcommand{\offer}{\mathit{offer}}
\newcommand{\C}{\mathsf{C}}
\newcommand{\CC}{\mathsf{CC}}
\newcommand{\CP}{\mathsf{CP}}

\section{Introduction}

Communications between software agents are typically regulated by interaction
protocols. These include general communication protocols, such as  the auction
protocol and the contract net protocol, as well as more specific protocols such as
the NetBill payment protocol \citep{mi:handbook03,mi:article:dunham}. 
Traditional protocol
representations such as Finite State Machines (FSM), Petri-Nets
\citep{mi:inproc:hmine} and AUML sequence diagrams 
often specify protocols in terms of legal message sequences.  Under such
protocol specifications, agent interactions are pre-defined and predictable. The
inevitable rigidity resulting from this prevents agents from taking
opportunities and handling exceptions in a highly dynamic and uncertain
multi-agent environment.


Commitment Machines proposed by \cite{mi:article:dunham} define an
interaction protocol in terms of actions that change the state of the system,
which consists of the state of the world as well as the {\it commitments} that
agents have made to each other. It is a commitment made to an interaction
partner which makes an agent perform its next action. In other words, an agent
acts because it wants to comply with the protocol and provide the promised
outcomes for another party. Actions not only change the values of state
variables, but also may initiate new commitments and/or discharge existing
commitments. In traditional protocol representations, agents are constrained to
perform a pre-defined sequence of actions, whereas in CMs, an agent is able to
reason about what action should be taken next in accordance with the dynamics of
the environment and the management of its commitments in that environment.  This
fundamentally changes the process of specifying a protocol from a procedural
approach (i.e.\ prescribing {\em how} an interaction is to be executed) to a
declarative one (i.e.\ describing {\em what} interaction is to take place).

Another advantage of the CM approach is that it provides a natural means of
managing multi-agent interactions. Agent programming concepts are often
discussed in the context of a single agent situated in an environment,
discussing properties such as autonomy, pro-activeness, reactivity and social
awareness. The CM approach enables pro-activeness and reactivity to be discussed
in a multi-agent context.

CMs thus allow interactions between agents to be organized in a manner which is
more flexible and robust than an approach based on pre-defined sequences. For
example, in the NetBill protocol, a
customer may wish to order goods without first receiving a quotation, or a
merchant may be happy to send goods to a known reliable customer with less
rigorous checking than normal.

In this thesis we identify a number of areas where the Commitment Machine
framework can be improved.  Specifically, we show how the identification of
undesirable states (such as omitting to provide a receipt, or receiving the
goods before payment has been confirmed) can be incorporated into the design
process in order to achieve acceptable outcomes for a wider variety of
circumstances than is done in \citep{mi:handbook03,mi:article:dunham}. 
We also show how certain
anomalies in discharging commitments and in handling pre-conditions can be
remedied. 

\section{Citation commands}

Citations in different styles. With the \texttt{natbib} package, 
you can use different
citation commands. 
The \texttt{\\citep} command gives author and year \citep{mi:handbook03},
The \texttt{\\cite} command gives author and year for use in text, 
for example \cite{mi:handbook03} wrote the data mining handbook.   



% Bibliography goes here
%\bibliographystyle{agsm}
\bibliographystyle{apalike}
\bibliography{DM}

% Keep these if you want appendices
%\newpage
%\appendix
%\section{An Appendix}

\end{document}


